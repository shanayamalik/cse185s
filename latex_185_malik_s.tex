% \documentclass{} precedes the preamble and is typically the first command in any .tex document. Every .tex document should include this command as it defines what kind of document you intend on creating. Modifiers in square brackets [] can be added in between the text ''documentclass'' and the curly brackets {} to modify font size, templates, etc. 

\documentclass[12pt,journal,compsoc]{IEEEtran}

%-----PACKAGES-------------------------------------------------------------------------------------

% Packages include extra commands that allow for additional formatting, ranging from Graphics, Math, to Alignment. The command to include packages will always look similar to \usepackage{} where the package name is within the curly brackets {}. Packages are defined in the preamble, i.e., between the \documentclass{} and \begin{document} commands.

\usepackage{graphicx}
\usepackage{textcomp}
\usepackage{graphicx}
\usepackage{amsmath}
\usepackage{hyperref}
\usepackage{booktabs}
\usepackage{geometry}

%----- The DOCUMENT Environment-------------------------------------------------------------------

% The \begin{document} and \end{document} commands establish the environment for the text of the document. The \begin{} and \end{} commands are used repeatedly in LaTeX to show where an environment begins and ends. The \end{document} command will be the last line of this .tex file.

\begin{document}

% The following commands are self explanatory. Insert your title, author and date into each command's curly bracket. You can include the abstract, paper header and paper footer information in this section then conclude the section with the command \maketitle (shown as the last line at the end of this section).

\title{\LaTeX\ Tutorial}
\author{Shanaya I. Malik}

\date{}		% leaving the brackets empty omits the date
% To input the current date, you can type: \date{\today}

% The paper headers
\markboth{\LaTeX\ Tutorial}%
{Moulds \MakeLowercase{\textit{et al.}}: CMPE185}
% The only time the second header will appear is for the odd numbered pages
% after the title page when using the twoside option.
% 
% *** Note that you probably will NOT want to include the author's ***
% *** name in the headers of peer review papers.                   ***
% You can use \ifCLASSOPTIONpeerreview for conditional compilation here if
% you desire.

% The publisher's ID mark at the bottom of the page is less important with
% Computer Society journal papers as those publications place the marks
% outside of the main text columns and, therefore, unlike regular IEEE
% journals, the available text space is not reduced by their presence.
% If you want to put a publisher's ID mark on the page you can do it like
% this:
\IEEEpubid{0000--0000/00\$00.00~\copyright~2007 IEEE}
% or like this to get the Computer Society new two part style.
%\IEEEpubid{\makebox[\columnwidth]{\hfill 0000--0000/00/\$00.00~\copyright~2007 IEEE}%
%\hspace{\columnsep}\makebox[\columnwidth]{Published by the IEEE Computer Society\hfill}}
% Remember, if you use this you must call \IEEEpubidadjcol in the second
% column for its text to clear the IEEEpubid mark (Computer Society jorunal
% papers don't need this extra clearance.)


% use for special paper notices
%\IEEEspecialpapernotice{(Invited Paper)}

\IEEEcompsoctitleabstractindextext{%
\begin{abstract}
%\boldmath
The tutorial offers an introduction to \LaTeX, a typesetting system that is particularly relevant within the academic community and often required for technical documentation. The key topics include handling of text, figures, tables, mathematical equations, and references.  The tutorial also emphasizes the importance of \LaTeX in ensuring document portability and the consistent representation of complex formats across various platforms.
\end{abstract}

% Note that keywords are not normally used for peerreview papers.
\begin{IEEEkeywords}
CSE185, \LaTeX\ Tutorial, IEEEtran, journal, \LaTeX, paper, template.
\end{IEEEkeywords}}

\maketitle

%----- The SECTION Environment -------------------------------------------------------------------

\section{Introduction}

\IEEEPARstart{T}{his} tutorial is designed to introduce novice engineering students to \LaTeX\, a software system for typesetting documents widely used in academia and the professional engineering world. The tool operates on the principle that the writer focuses primarily on the content, using plain text and commands to control formatting.

\subsection{Why is this Tutorial Helpful?}
\LaTeX\ is meant for managing complex mathematical equations, tables, and figures, making it an ideal choice for technical documents. The tutorial will guide students through the process of creating documents that are visually professional and precisely formatted.

\subsubsection{Why Should I Learn \LaTeX{}?}
If students are familiar with \LaTeX, it will allow for the development of well-structured scientific reports and papers. \LaTeX\ documents maintain consistency across different computing platforms, improving project collaboration. 

%---- Additional Features ----
\section{Creating a \LaTeX\ File}
\IEEEPARstart{I}{n} order to create a '.tex' file, it is required to use a text editor, but no specialized software is needed. Editors, such as Overleaf, include \LaTeX\ syntax highlighting and integration.

\subsection{Environments}
Environments in \LaTeX\ determine text formatting, and are represented as \texttt{\textbackslash begin} and \texttt{\textbackslash end} commands. It can be used for various purposes, such as itemizing lists, aligning text, or formatting mathematical expressions.

\subsubsection{Begin and End}
The \texttt{\textbackslash begin\{environment\}} command initiates the environment, where \texttt{environment} is the name of the environment you wish to use.  The \texttt{\textbackslash end\{environment\}} command signals the end of an environment. \LaTeX\ will stop applying the special formatting of the environment after this command.  

\subsection{Reserved Characters}
In \LaTeX\, there are some characters have special meanings and are 'reserved' for specific commands and functions. The characters include symbols, such as the backslash `\`, which indicates the start of a command, and the percent `\%`, which comments out text, to name a few. In order to display these characters as part of your text, the character must be escaped using specific commands. 

\subsubsection{Why are these Reserved?}
There are reserved characters in \LaTeX\, which serve specific functions in the typesetting process, such as invoking commands or formatting text. The characters cannot be used directly to display the literal symbols without being escaped.

\subsubsection{Explain the Functions of the Characters: \textbackslash , \textasciitilde , \textbackslash\textbackslash , \%}

\begin{itemize}
    \item \textbackslash: Introduces a command or a \LaTeX keyword.
    \item \textasciitilde: Produces a non-breakable space.
    \item \textbackslash\textbackslash: Creates a new line.
    \item \% : Starts a comment.
\end{itemize}

\subsubsection{What if you Want to Display these Characters?}
In order to display reserved characters in the text, a specific set of commands tells \LaTeX to treat those as regular characters and not as commands or special symbols.

\begin{itemize}
    \item To display a backslash, use \texttt{\textbackslash textbackslash}.
    \item To display a tilde, use \texttt{\textbackslash textasciitilde} or place it inside math mode as \texttt{\$ \textasciitilde \$}.
    \item To display double backslashes, use \texttt{\textbackslash textbackslash\textbackslash textbackslash}.
    \item To display the percent sign, use \texttt{\textbackslash textpercent}.
\end{itemize}

\subsection{Preamble}
The preamble of a \LaTeX document is the section where you define the global settings and include the necessary packages that affect the entire document. It starts right after the \textbackslash documentclass command and continues until the \textbackslash begin\{document\} command. The preamble sets up elements such as the document type, packages, custom commands, and other configurations needed.

\subsubsection{\textbackslash documentclass[] and class files}
The \textbackslash documentclass[] command is the first line in a \LaTeX\ file, and it defines the layout and style of the document. The options within the square brackets set global parameters such as font size, paper size, and whether the document is one-sided or two-sided. There are also class files, which are predefined settings that format the document according to the type of content. 

\subsubsection{Packages}

Packages in \LaTeX\ are extensions that add functionality to the typesetting system. The packages are included in the preamble using the \textbackslash usepackage\{name\} command and provide additional fonts and macros or enable the document to handle complex tasks, such as advanced mathematical formatting, inserting images, handling links, or modifying the layout. 

\subsubsection{Begin and End Document}
The \textbackslash begin\{document\} command marks the beginning of the document's content section. If there is any text before this command, it is considered to be a part of the preamble.  In comparison, the \textbackslash end\{document\} command is used to signify the end of the document. If there is anything written after this command, it will not be processed.

\subsection{Title and Heading Information}
The title, author, and date are elements declared in the preamble and displayed in the document through a specific command.

\subsubsection{\textbackslash maketitle command}
If the title, author, and date have been set in the preamble, it can use the `\maketitle` command within the document body to generate a title page with this information. The `\textbackslash{}maketitle` command uses the information provided by `\textbackslash{}title`, `\textbackslash{}author`, and `\textbackslash{}date` to format a standard title page according to the document class being used. The `\textbackslash{}maketitle` command is flexible and will automatically adjust the formatting of the title page based on the document class and any packages that might influence title page formatting. 

\section{Sections}
\IEEEPARstart{I}{n} a \LaTeX\ document, sections are major divisions and are created using the \textbackslash{}section command. This command numbers the sections and includes it in the document’s Table of Contents.

\subsection{Subsections}
There are subsections, which are used to create smaller divisions within a section. In \LaTeX\, subsections are created using the \textbackslash{}subsection command.

\section{Body Text: Paragraphs and Content}
\IEEEPARstart{T}{he} body text of a document is essential as it includes the main message(s) and information. In a \LaTeX text document, paragraphs are fundamental, and created by leaving a blank line between two blocks of text. The behavior can be altered by changing the length of \texttt{\textbackslash parindent} for indentation and \texttt{\textbackslash parskip} for vertical spacing between paragraphs.  

\subsubsection{Indentation and Spacing}
The text in \LaTeX\ is justified as the default, meaning that the lines stretch to be left-aligned (\texttt{l}), centered (\texttt{c}), or right-aligned (\texttt{r}).

\subsection{Content Blocks and Quotations}
The \texttt{quote} and \texttt{quotation} environments can be used. These provide a visual distinction from the rest of the text, indenting the content from the main body.

\section{Tables}
\IEEEPARstart{T}{he} creation and management of tables in \LaTeX\ involve a combination of the \texttt{table} and \texttt{tabular} environments.

\subsection{Difference Between Table and Tabular Environments}
The \texttt{\textbackslash begin\{table\}} environment is used to add a floating table structure, which can include a caption or a label for referencing. This is particularly useful when the document's layout and formatting require the table not to disrupt the text. The \texttt{\textbackslash begin\{tabular\}} environment is where the table data is inputted and formatted. 

\subsection{Characters and Symbols in Tables}
In the \texttt{tabular} environment, the following symbols and characters are often used:
\begin{itemize}
    \item \texttt{l}, \texttt{c}, \texttt{r} represent column text alignment (left, center, right).
    \item \texttt{|} represents a vertical line between columns.
    \item \texttt{\textbackslash hline} is used to insert a horizontal line between rows.
    \item \texttt{\textbackslash\textbackslash} ends a row and starts a new one.
\end{itemize}

\subsection{Entering Content}
In order to enter content into a table, each cell is separated by an ampersand (\texttt{\&}), and each row is concluded with a double backslash (\texttt{\textbackslash\textbackslash}). 

\subsection{Creating a Table in \LaTeX}
The following is an example of a basic table in \LaTeX\, using the \texttt{table} and \texttt{tabular} environments.

\begin{enumerate}
    \item Please begin the floating table environment with \texttt{\textbackslash begin\{table\}} and center it using \texttt{\textbackslash centering}.
    \item Include a caption with \texttt{\textbackslash caption\{...\}} and a label with \texttt{\textbackslash label\{...\}} for referencing.
    \item Start the tabular environment with \texttt{\textbackslash begin\{tabular\}\{...\}} specifying column alignments.  
    \item Define the header row using bold text (\texttt{\textbackslash textbf\{...\}}) and separate columns with \texttt{\&}. Terminate rows with \texttt{\textbackslash\textbackslash}.
    \item Close both the \texttt{tabular} and \texttt{table} environments with \texttt{\textbackslash end\{tabular\}} and \texttt{\textbackslash end\{table\}} respectively.
\end{enumerate}

\begin{table}[h]
\centering
\caption{An Example Table}
\label{tab:example}
\begin{tabular}{|l|c|}
\hline
\textbf{Left-Aligned Header} & \textbf{Centered Header} \\ \hline
Data 1                       & Analysis 1            \\ \hline
Data 2                       & Analysis 2            \\ \hline
Data 3                       & Analysis 3            \\ \hline
\end{tabular}
\end{table}

\section{Figures}
\IEEEPARstart{F}{igures} are an important aspect of scientific documentation, as it allows for the visual representation of data.  

\subsection{Including the Graphicx Package}
The \texttt{graphicx} package is relevant for managing images in \LaTeX. It should be included in the preamble, enabling commands such as \texttt{\textbackslash includegraphics}. The following line can be added at the beginning of the \LaTeX document, before \texttt{\textbackslash begin\{document\}}:
\begin{verbatim}
\usepackage{graphicx}
\end{verbatim}

\subsection{Inserting a Figure}
In order to add a figure to your document, use the \texttt{figure} environment. The environment allows the writer to include images, provide captions, and ensure figures are correctly positioned.

\begin{itemize}
  \item The figure environment with placement specifiers is started with \texttt{\textbackslash begin\{figure\}[ht]}. 
  \item The \texttt{\textbackslash centering} command centers the figure in the text column.
  \item In order to insert the image, the command, \texttt{\textbackslash includegraphics\{image.png\}}, should be used, where the path is replaced with the image file path. 
  \item The following command will allow the writer to insert a caption, \texttt{\textbackslash caption\{...\}}.
  \item In the text, the figure can be referenced with \texttt{\textbackslash label\{...\}}.
\end{itemize}

\subsubsection{Inserting a Titration Plot}
\begin{figure}[htbp]
  \centering
  \includegraphics[width=\linewidth]{Titration_Curve.png}
  \caption{Table of pH Measurements at Incremental Volumes of Base Added During a Titration Process}
  \label{fig:titrationplot}
\end{figure}

\subsection{Creating Graphs}
In \LaTeX, graphs can be created using the \texttt{pgfplots} package, or can include a pre-made graph using \texttt{\textbackslash includegraphics}.

\subsection{Including a Meaningful Caption}
The usage of captions is that it provides context and explain the significance of the figure or graph, which can be initiated with \texttt{\textbackslash caption} and followed with the description.

\subsection{Referencing the Graph in Text}
In order to reference the figure in the text, it can be done using the \texttt{\textbackslash ref} command with the label assigned to the figure.

\section{Mathematical Formulas}
\IEEEPARstart{M}{athematical} formulas are central to engineering and scientific documents. \LaTeX provides a system for typesetting complex mathematical equations and symbols.  

\subsection{Equation Environments}
\LaTeX uses two primary methods for displaying equations: inline and display. Inline equations appear within the text, while display equations are separated from the main text.

\subsection{Inline and Display Equations}
Insert inline equations using dollar signs (\$). The display equations are separated from the text and often centered. The `equation` environment is used for that purpose.

\begin{verbatim}
\begin{equation}
    E = mc^2
\end{equation}
\end{verbatim}

It will appear in the \LaTeX document like this:

\begin{equation}
    E = mc^2
\end{equation}

\section{Mathematical Formulas}

\subsection{Symbols}
\LaTeX supports an extensive array of mathematical symbols, including Greek letters \( \delta \); summation symbols \( \sum \); and integral symbols \( \int \).

\subsection{Fractions}
Fractions in LaTeX are created with the \textbackslash frac{} command, which takes the numerator and the denominator as arguments. 

\begin{verbatim}
\frac{numerator}{denominator}
\end{verbatim}

\subsubsection{Writing the Fraction One-Half}
In order to write one-half as a fraction, the following command should be used in the \LaTeX document. 
\begin{verbatim}
\frac{1}{2}
\end{verbatim}

This will appear as \( \frac{1}{2} \). 

\subsection{Complex Equations}
In order to write a complicated function in \LaTeX, use an array environment to organize the parameters of the function vertically and horizontally. 

\subsubsection{Hypergeometric Function}
The empty curly braces at the beginning are used to correctly align the subscripts and superscripts before the function symbol.

\begin{verbatim}
{}_{3}F_{2}\left
[\begin{array}{ccc}
a, & b, & c; \\
   & d, & e; 
\end{array} z\right]
\end{verbatim}

The code will produce the following hypergeometric function after compilation:

\begin{equation}
{}_{3}F_{2}\left[
\begin{array}{ccc}
a, & b, & c; \\
   & d, & e; 
\end{array} z\right]
\end{equation}

\subsubsection{Binomial Function}
The binomial coefficient is can be written in \LaTeX using the \textbackslash frac command for fractions or the \textbackslash binom command for a more compact notation.  

\begin{verbatim}
\frac{n!}{k!(n-k)!} 
= \binom{n}{k}
\end{verbatim}

This code will display the binomial coefficient like the following, when compiled:
\begin{equation}
\frac{n!}{k!(n-k)!} = \binom{n}{k}
\end{equation}

\subsection{Matrices}
A 2x2 matrix would be written as the following. 

\begin{verbatim}
\begin{equation}
\begin{matrix}
a & b \\
c & d \\
\end{matrix}
\end{equation}
\end{verbatim}

\section{How to: Acknowledgments}
\IEEEPARstart{T}{he} acknowledgments section is an essential part of any academic paper or report. It is a space where the author can thank those who have contributed to the work but are not among the authors, including advisors, funding agencies, technical staff, or colleagues. In \LaTeX, an acknowledgments section can be included by using the \texttt{\textbackslash section*\{Acknowledgments\}} command allows the writer to create a non-numbered section. The asterisk (*) next to \texttt{\textbackslash section} prevents the section number from appearing next to the title.

\subsection*{Placement of Acknowledgments}
In a document, the acknowledgments should ideally appear after the conclusion and before the references. This placement ensures that readers can easily find the acknowledgments but does not disrupt the flow of the paper's main content.

\section{How to: References}
\IEEEPARstart{M}{anaging} references in \LaTeX{} is efficient and allows for consistent citation formatting.  

\subsection{Thebibliography Environment}
Listing references can be done within the \texttt{thebibliography} environment, where each entry is defined by \texttt{\textbackslash bibitem}. 

\subsection{Using \textbackslash label and \textbackslash ref}
In order to assign a label to figures, tables, or sections, please use the \texttt{\textbackslash label} command and reference it using \texttt{\textbackslash ref}.

\subsection{Citing References}
For citations that are within the main text, use the \texttt{\textbackslash bibitem} command to illustrate the source. 

\subsection{Compiling the Document}
The document should be compiled twice to process references, as initially compiling it once can result in information not presenting correctly. The first pass collects the reference keys, and the second pass updates the citations.

\section{Conclusion}
\IEEEPARstart{T}{he} use and flexibility of \LaTeX allows for an improved form of technical documentation. The typesetting program's capabilities allow any engineer to become better prepared for academic and professional communication. 

\section*{Acknowledgments}
\IEEEPARstart{T}{he} author would like to express gratitude to the individuals and resources that were helpful in the completion of this assignment. Thank you to Kevin Bowden, a Teaching Assistant of UCSC's CSE 185S. Professor Gerald Moulds provided insightful instruction, which has improved the author's understanding and application of technical communication principles. The creators of Overleaf have provided a comprehensive tutorial that was read and appreciated for the clear explanation and practical examples.

\begin{thebibliography}{3}
\bibitem{OverleafTutorials}
Overleaf. (n.d.). \textit{Learn LaTeX in 30 minutes}. Retrieved from \url{https://www.overleaf.com/learn/latex/Tutorials}

\bibitem{LatexProject}
LaTeX Project. (n.d.). \textit{LaTeX – A Document Preparation System}. Retrieved from \url{https://www.latex-project.org/}

\bibitem{StackExchangeLatex}
Stack Exchange. (n.d.). \textit{TeX - LaTeX Stack Exchange}. Retrieved from \url{https://tex.stackexchange.com/}

\end{thebibliography}

\end{document}
