\documentclass[12pt,journal,compsoc]{IEEEtran}

%-----PACKAGES-------------------------------------------------------------------------------------

\usepackage{graphicx}        % For including graphics
\usepackage{amsmath}         % For mathematical formulas
\usepackage{cite}            % For citations
\usepackage{hyperref}        % For hyperlinks
\usepackage{array}           % For better arrays (e.g., matrices) in maths
\usepackage{caption}         % For customizing captions
\usepackage{subcaption}      % For subfigures
\usepackage{booktabs}        % For better table formatting
\usepackage{multicol}        % For multi-column text
\usepackage{enumitem}        % For customizable list environments
\usepackage{float}           % For improved float control
\usepackage{lipsum}          % For placeholder text

%----- The DOCUMENT Environment-------------------------------------------------------------------

\begin{document}

\title{Literature Review on Adaptive Telemetry Systems for Monitoring Neonatal Neurodevelopment}
\author{Shanaya I. Malik}

\date{\today}

% The paper headers
\markboth{Literature Review on Adaptive Telemetry Systems for Monitoring Neonatal Neurodevelopment}%

\IEEEcompsoctitleabstractindextext{%
\begin{abstract}
%\boldmath
The literature review explores the current practical applications and future advancements of adaptive telemetry systems for monitoring neonatal neurodevelopment. The key systems that are mentioned include Amplitude-Integrated EEG (aEEG) and Near-Infrared Spectroscopy (NIRS), both providing continuous, real-time data for early diagnosis and intervention of neurological conditions in neonates. The integration of aEEG and NIRS offers a comprehensive approach, combining electrical brain activity monitoring with cerebral oxygenation data. The review also highlights the potential of portable EEG monitoring, as well as offers insight into research that can improve the accuracy, reliability, and accessibility of said monitoring systems to better support neonatal neurodevelopment.
\end{abstract}

\begin{IEEEkeywords}
Adaptive, telemetry, systems, neonatal, neurodevelopment. 
\end{IEEEkeywords}}

\maketitle

\section{Current Practical Applications of Adaptive Telemetry Systems for Monitoring Neonatal Brain Development}

\IEEEPARstart{A}{daptive} telemetry systems, such as amplitude-integrated EEG (aEEG) and Near Infrared Spectroscopy (NIRS), are important for monitoring neonatal brain development. The systems provide continuous, real-time data that aid in the early diagnosis and intervention of neurological conditions in neonates, a newborn that is of age four weeks or less. 

\subsection{Amplitude-Integrated EEG (aEEG)}

\subsubsection{Overview of aEEG}
The amplitude-integrated EEG (aEEG) has become a standard tool in neonatal neurological care. Initially developed in 1960 for adults, it was later modified for neonates to provide continuous monitoring of brain function. The aEEG is particularly useful for detecting seizures and assessing brain activity in neonates with hypoxia-ischemia and other neurological conditions \cite{IEEEhowto:toet}.

\subsubsection{aEEG Applications in Neonatal Care}
The aEEG is routinely used in neonatal intensive care units (NICUs) for several applications, including, but not limited to, the following. 

\begin{itemize}
    \item \textbf{Detection of Seizures:} aEEG provides continuous monitoring that can detect both clinical and subclinical seizures.
    \item \textbf{Assessment of Brain Function Post-Hypoxia:} aEEG is used to evaluate the brain function of neonates who have experienced perinatal asphyxia.
    \item \textbf{Monitoring in Preterm Infants:} aEEG is also valuable in monitoring the brain function of preterm infants.
\end{itemize}

\subsubsection{Technical Considerations and Challenges}
The use of aEEG in clinical practice can include challenges, as artefacts such as ECG interference and movement artefacts can lead to misinterpretations. The integration of EEG alongside aEEG in newer devices helps mitigate issues, allowing for more accurate interpretation of brain activity \cite{IEEEhowto:toet}.

\subsection{Near Infrared Spectroscopy (NIRS)}

\subsubsection{Overview of NIRS}
Near Infrared Spectroscopy (NIRS) is a non-invasive method for monitoring tissue oxygenation and cerebral hemodynamics. NIRS measures regional cerebral oxygen saturation (rScO2), providing valuable data on the brain's oxygen supply. 

\subsubsection{Technical Considerations and Challenges}
NIRS faces challenges, primarily related to movement artefacts and the need for proper fixation of the transducer. 

\subsection{Integration of aEEG and NIRS}
The combination of aEEG and NIRS provide a comprehensive approach to neonatal brain monitoring, as aEEG focuses on electrical brain activity and NIRS provides data on cerebral oxygenation. The integration of the two enhances the ability to monitor and manage neonatal neurodevelopmental issues. 

\section{Neonatal Intensive Care Monitoring}
Neonatal intensive care monitoring is relevant for maintaining physiological parameters within safe limits, especially in critically ill infants.  

\subsection{Current Monitoring Practices}
\subsubsection{Overview}
Monitoring physiological parameters such as heart rate, respiratory rate, blood gas levels, and oxygen saturation is integral to neonatal intensive care. 

\subsubsection{Relevant Practices}
The current monitoring practices include the use of pulse oximetry, transcutaneous monitoring, and intra-arterial sensors. 

\begin{itemize}
    \item \textbf{Pulse Oximetry:} It is widely used for monitoring oxygen saturation (SaO2). 
    \item \textbf{Transcutaneous Monitoring:} The transcutaneous monitors measure carbon dioxide (TcCO2) and oxygen (TcO2) levels.  
    \item \textbf{Intra-Arterial Sensors:} The sensors offer precise and continuous measurements of blood gas parameters. 
\end{itemize}

\subsection{Potential Improvements to Monitoring Systems}

\subsubsection{Biocompatibility}
Improving the catheters and sensors is important for reducing complications, such as blockage and infection. The advanced materials and coatings that mimic natural cell membranes can significantly enhance the performance and lifespan of such devices \cite{IEEEhowto:nicklin}.

\subsubsection{Signal-Processing}
Enhanced signal-processing technologies, such as adaptive filtering, can improve the accuracy of monitoring devices by reducing artifacts caused by motion and low perfusion. This leads to more reliable and precise measurements.

\subsubsection{Telemetry}
The telemetry systems allow for the wireless transmission of monitoring data, reducing the need for multiple physical connections to infants. Th technology facilitates easier handling of neonates and enhances mobility, particularly during transport or routine care activities \cite{IEEEhowto:nicklin}.

\subsubsection{Developmental Techniques}

\paragraph{Near-Infrared Spectroscopy (NIRS)}
NIRS is used to monitor tissue oxygenation and haemodynamics non-invasively. It measures changes in haemoglobin saturation and concentration, providing valuable insights into cerebral oxygenation and central blood volume. 

\paragraph{Peripheral Oxygen Consumption}
Measuring peripheral oxygen consumption (VO2) using NIRS can indicate early circulatory compromise. The metric is important for assessing the oxygen delivery and utilization status in neonates.

\subsection{Risk Management}
Introducing new monitoring technologies in neonatal care involves significant risk management considerations. Ensuring that new devices meet safety and performance standards is crucial \cite{IEEEhowto:nicklin}.

\section{Portable Neonatal EEG Monitoring and Sonification on an Android Device}
he implementation of a portable, low-cost EEG monitoring system using smartphone technology, which addresses the limitations and provide real-time brain health assessment through EEG sonification.

\subsection{EEG Monitoring and Sonification}
Electroencephalogram (EEG) monitoring is essential for understanding and tracking the electrical activity in the brain. The development of smartphone-based systems in collaboration with portable EEG acquisition devices offers a cost-effective and portable solution for EEG monitoring \cite{IEEEhowto:poveda}.

\subsubsection{Benefits of Sonification}
The sonification of EEG data involves converting the EEG signals into audible sound, which can be more intuitive for real-time monitoring compared to visual interpretation. The method leverages the human ear's superior ability to assess both spatial and temporal changes in frequency characteristics, allowing faster and more efficient monitoring of neonatal brain health \cite{IEEEhowto:poveda}.

\subsection{Implementation of Portable EEG Monitoring}
The system includes an EEG acquisition device that transmits data to an Android smartphone via Bluetooth Low Energy (BLE). It is preferred for its low power consumption and sufficient data transfer speeds, making it ideal for continuous monitoring applications.

\subsubsection{Multi-Threaded Processing}
In order to achieve real-time performance, the system uses multi-threaded processing. The vocoder's workload is distributed across multiple threads, allowing the operating system to optimize core usage.  

\subsection{Results and Discussion}
The system was tested on a Samsung Galaxy Grand Prime smartphone, demonstrating the relationship between real-time EEG processing and sonification. The results showed that using multiple threads significantly improved the system's performance, allowing it to handle the computational load effectively while maintaining low power consumption.  

\subsubsection{Practical Implications}
The portable EEG monitoring system offers practical benefits, including it providing a low-cost alternative to traditional EEG monitoring equipment. The compact and mobile nature of the system makes it accessible for use in various settings, including remote and underserved areas. The implementation of a portable EEG monitoring and sonification system on an Android device highlights the cost-effectiveness, portability, and real-time capabilities of improving the detection and management of neonatal brain injuries.

\begin{thebibliography}{1}

\bibitem{IEEEhowto:toet}
M.~C.~Toet and P.~M.~A.~Lemmers, “Brain Monitoring in Neonates,” \emph{Web of Science}, accessed May 14, 2024. [Online]. Available: \url{https://www.webofscience.com/wos/woscc/full-record/WOS:000263512700003}

\bibitem{IEEEhowto:nicklin}
S.~Nicklin, A.~et al., “Neonatal Intensive Care Monitoring,” \emph{Current Paediatrics}, accessed May 14, 2024. [Online]. Available: \url{https://www.sciencedirect.com/science/article/pii/S0957583903001283}

\bibitem{IEEEhowto:poveda}
J.~Poveda, M.~O'Sullivan, E.~Popovici, and A.~Temko, "Portable neonatal EEG monitoring and sonification on an Android device," in \emph{Proc. 2017 39th Annual International Conference of the IEEE Engineering in Medicine and Biology Society (EMBC)}, Jeju, Korea (South), 2017, pp. 2018-2021, doi: 10.1109/EMBC.2017.8037248.

\end{thebibliography}

\end{document}
