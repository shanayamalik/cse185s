\documentclass[12pt,journal,compsoc]{IEEEtran}

%-----PACKAGES-------------------------------------------------------------------------------------

\usepackage{graphicx}        % For including graphics
\usepackage{amsmath}         % For mathematical formulas
\usepackage{cite}            % For citations
\usepackage{hyperref}        % For hyperlinks
\usepackage{array}           % For better arrays (e.g., matrices) in maths
\usepackage{caption}         % For customizing captions
\usepackage{subcaption}      % For subfigures
\usepackage{booktabs}        % For better table formatting
\usepackage{multicol}        % For multi-column text
\usepackage{enumitem}        % For customizable list environments
\usepackage{float}           % For improved float control
\usepackage{lipsum}          % For placeholder text

%----- The DOCUMENT Environment-------------------------------------------------------------------

\begin{document}

\title{Literature Review on Adaptive Telemetry Systems for Monitoring Neonatal Neurodevelopment}
\author{Shanaya I. Malik}

\date{\today}

% The paper headers
\markboth{Literature Review on Adaptive Telemetry Systems for Monitoring Neonatal Neurodevelopment}%

\IEEEcompsoctitleabstractindextext{%
\begin{abstract}
%\boldmath
The abstract goes here.
\end{abstract}

\begin{IEEEkeywords}
Adaptive, telemetry, systems, neonatal, neurodevelopment. 
\end{IEEEkeywords}}

\maketitle

%----- The SECTION Environment -------------------------------------------------------------------

\section{Current Practical Applications of Adaptive Telemetry Systems for Monitoring Neonatal Brain Development}

\IEEEPARstart{A}{daptive} telemetry systems, such as amplitude-integrated EEG (aEEG) and Near Infrared Spectroscopy (NIRS), are important for monitoring neonatal brain development. The systems provide continuous, real-time data that aid in the early diagnosis and intervention of neurological conditions in neonates, a newborn that is of age four weeks or less. 

\subsection{Amplitude-Integrated EEG (aEEG)}

\subsubsection{Overview of aEEG}
The amplitude-integrated EEG (aEEG) has become a standard tool in neonatal neurological care. It was developed in 1960 for adults and was later modified for neonates with the intention of providing continuous monitoring of brain function. The aEEG is particularly useful for detecting seizures and assessing brain activity in neonates with hypoxia-ischemia and other neurological conditions \cite{IEEEhowto:toet}.

\subsubsection{Practical Applications in Neonatal Care}
The aEEG is routinely used in neonatal intensive care units (NICUs) for several applications, including but not limited to the following. 

\begin{itemize}
    \item \textbf{Detection of Seizures:} aEEG provides continuous monitoring that can detect both clinical and subclinical seizures.  
    \item \textbf{Assessment of Brain Function Post-Hypoxia:} aEEG is used to evaluate the brain function of neonates who have experienced perinatal asphyxia.  
    \item \textbf{Monitoring in Preterm Infants:} aEEG is also valuable in monitoring the brain function of preterm infants.  
\end{itemize}

\subsubsection{Technical Considerations and Challenges}
The use of aEEG in clinical practice is not without challenges. Artefacts such as ECG interference and movement artefacts can lead to misinterpretations. However, the integration of "real" EEG alongside aEEG in newer devices helps mitigate these issues, allowing for more accurate interpretation of brain activity \cite{IEEEhowto:toet}.

\subsection{Near Infrared Spectroscopy (NIRS)}

\subsubsection{Overview of NIRS}
Near Infrared Spectroscopy (NIRS) is a non-invasive method for monitoring tissue oxygenation and cerebral hemodynamics. NIRS is increasingly being used in neonatal care to provide continuous monitoring of brain oxygenation \cite{IEEEhowto:toet}.

\subsubsection{Practical Applications in Neonatal Care}
NIRS has several practical applications in the NICU:

\begin{itemize}
    \item \textbf{Monitoring Cerebral Oxygenation:} NIRS measures regional cerebral oxygen saturation (rScO2), providing valuable data on the brain's oxygen supply. This information is crucial for managing infants with conditions like hypoxia-ischemia, patent ductus arteriosus (PDA), and during high mean airway pressure ventilation \cite{IEEEhowto:toet}.
    \item \textbf{Guiding Interventions:} By monitoring cerebral oxygen imbalance, NIRS can guide clinical interventions to correct this imbalance, thereby improving patient outcomes. For example, during cardiac surgery or post-surgery, continuous NIRS monitoring helps ensure that the brain is adequately oxygenated \cite{IEEEhowto:toet}.
\end{itemize}

\subsubsection{Technical Considerations and Challenges}
Similar to aEEG, NIRS also faces challenges, primarily related to movement artefacts and the need for proper fixation of the transducer. Despite these challenges, NIRS has proven to be a reliable method for long-term monitoring of cerebral oxygenation in critically ill neonates \cite{IEEEhowto:toet}.

\subsection{Integration of aEEG and NIRS}
Combining aEEG and NIRS provides a comprehensive approach to neonatal brain monitoring. While aEEG focuses on electrical brain activity, NIRS provides data on cerebral oxygenation. The integration of these two modalities enhances the ability to monitor and manage neonatal neurodevelopmental issues, offering a more robust and holistic view of the neonate's brain health \cite{IEEEhowto:toet}.

\subsection{Conclusion}
Adaptive telemetry systems such as aEEG and NIRS are invaluable tools in neonatal care, offering continuous, real-time monitoring of brain function and oxygenation. These systems play a crucial role in early detection, diagnosis, and management of neurological conditions in neonates, thereby improving neurodevelopmental outcomes. Despite some technical challenges, the advancements and integration of these technologies continue to enhance their clinical utility and effectiveness.


\section{Conclusion}
Conclusion goes here.

%----- APPENDICES --------------------------------------------------------------------------------
\appendices
\section{Appendix Title}
Appendix one text goes here.

% you can choose not to have a title for an appendix
% if you want by leaving the argument blank
\section{}
Appendix two text goes here.


%----- ACKNOWLEDGEMENT SECTION -------------------------------------------------------------------
% Explain what the asterisk * does in the next line: 
\section*{Acknowledgements}

The author would like to thank...\\ \\

\begin{thebibliography}{1}

\bibitem{IEEEhowto:toet}
M.~C.~Toet and P.~M.~A.~Lemmers, “Brain Monitoring in Neonates,” \emph{Web of Science}, accessed May 14, 2024. [Online]. Available: \url{https://www.webofscience.com/wos/woscc/full-record/WOS:000263512700003}

\bibitem{IEEEhowto:nicklin}
S.~Nicklin, A.~et al., “Neonatal Intensive Care Monitoring,” \emph{Current Paediatrics}, accessed May 14, 2024. [Online]. Available: \url{https://www.sciencedirect.com/science/article/pii/S0957583903001283}

\bibitem{IEEEhowto:poveda}
J.~Poveda, M.~O'Sullivan, E.~Popovici, and A.~Temko, "Portable neonatal EEG monitoring and sonification on an Android device," in \emph{Proc. 2017 39th Annual International Conference of the IEEE Engineering in Medicine and Biology Society (EMBC)}, Jeju, Korea (South), 2017, pp. 2018-2021, doi: 10.1109/EMBC.2017.8037248.

\end{thebibliography}

\end{document}
